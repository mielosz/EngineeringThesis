\chapter{Podsumowanie}
\section{Weryfikacja założeń projektowych}
Odnosząc się do zakresu pracy z Rozdziału 1. można stwierdzić, że wszystkie z~postanowionych założeń zostały spełnione na poziomie zadowalającym.

Na szczególną uwagę zasługuje wykonanie bloku filtra adaptacyjnego.
Stworzono element, który do tej pory nie był dostępy w środowisku \texttt{Gnuradio}.
Na etapie implementacji wystąpiły trudności przetwarzaniem próbek zespolonych, co ograniczyło możliwości filtracji do sygnałów rzeczywistych.

Skuteczna symulacja warunków pracy stworzyła w pełni kontrolowane i hermetyczne środowisko, co pomogło w wykazaniu efektu filtracji.
Udowodniono, że system eliminacji zakłóceń polepsza parametry transmisji w zaszumionych systemach, zatem cel pracy został osiągnięty.



\section{Możliwości rozwoju projektu}
Niniejsza praca prezentowała zastosowanie filtracji z przetwarzaniem odbywającym się po stronie komputera PC. Był to pewnego rodzaju prototyp, więc aby odbiornik z takim systemem znalazł zastosowanie w kompaktowych, współczesnych rozwiązaniach telekomunikacyjnych, należy rozważyć zintegrowanie go z wbudowanym komputerem lub zewnętrznym miniaturowym komputerem z dedykowanymi procesorami do przetwarzania sygnałów. 