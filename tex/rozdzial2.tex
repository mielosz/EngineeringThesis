\chapter{Filtr adaptacyjny}
\section{Implementacja}
\subsection{Środowisko GnuRadio}
\subsection{Filtracja próbek rzeczywistych}
\subsection{Filtracja próbek zespolonych}
Sygnał zespolony możemy zapisać jako sumę dwóch sygnałów rzeczywistych:
\begin{equation}
u(n) = u_I(n) + ju_Q(n)
\end{equation}
gdzie $u_I(n)$ oznacza składową rzeczywistą sygnału, a $u_Q(n)$ składową urojoną.
\section{Blok funkcjonalny}
\subsection{Założenia}
\paragraph{}
Pierwszym założeniem projektowym bloku było zapewnienie interfejsu zgodnego ze schematem filtra. 
Powinien posiadać dwa wejścia. 
Dla sygnału zakłóconego i sygnału odniesienia. Wyjścia również dwa. 
Pierwsze- sygnału odfiltrowanego, drugie- wektora błędu do monitorowania szybkości zbieżności algorytmu.

Kolejny parametr określa liczbę próbek wyjściowych do wejściowych.
Blok nie będzie zmieniał liczby próbek sygnału wejściowego zatem stosunek $n_{wej}/n_{wyj}$
powinien wynosić $1:1$. W nomenklaturze środowiska GnuRadio takie przetwarzanie określa się jako synchroniczne.

Typem danych wejściowych jak i wyjściowych będą wartości zespolone próbek.
Parametrami do ustawienia przez użytkownika w środowisku graficznym będą:
\begin{enumerate}
 \item rząd $N$ filtru w postaci liczby całkowitej 
 \item długość kroku $\mu$ w postaci liczby zmiennoprzecinkowej
 \item liczba iteracji w postaci liczby całkowitej
 \item początkowe współczynniki filtra w postaci wektora liczb o długości N 
 \end{enumerate}
 
Parametry zostaną podzielone na wymagane (1, 2) i opcjonalne (3, 4)